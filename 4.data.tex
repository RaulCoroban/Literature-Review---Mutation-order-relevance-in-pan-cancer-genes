
Causal or non-causal chronological order requires a convoluted model to accurately explain most of interactions causes and effects. Given trials observations, the system could be portrayed solely

\subsection{Chronological order of mutation impact disease evolution}
Colorectal cancer is one of the best approaches for studying cancer evolution. They allow a full display of genetic alterations which eventually develop a neoplasm. Besides, the availability of differential scenarios such as benignant and malignant states of the disease and their study (Sugarbaker et al., 1985), allows for full genomic progression analysis. [ref] show a model which scrutinises colorectal neoplasia features, from oncogene activation - tumour-suppressor inactivation (benignant state) to phenotypic effects, visible at the last stages of the disease.
Via somatic alterations analysis performed at various stages of CRC formation, clonal expansion has been identified as a common feature, both for this specific tumour and others. The clonal nature confers growth advantage to the neoplasm in order to become the predominant cell type [ref]. We should consider this event as common in the neoplasia field, rather than specific to CRC.
\\

Furthermore, in this particular case, mutations occur at characteristic stages of the progression. E.g. 17p events of deletion occurs in late phases, and 5q and 18q loss happen afterwards. However, order preference does not set order restriction. Even if 17p deletion is bound to a stage of the disease, its occurrence might be feasible in any, probably at a lower rate. Besides, this preferred order is coupled with progressive accumulation, blurring the impact each of the cases has on the progression. By way of explanation (using the same example above), 17p loss can be found in early-stage scenarios of CRC, and this event is accumulated to other alterations to raise a new stage,  therefore, accumulation has a higher impact than chronological order. On the other hand, its mere occurrence “before its preferred stage” might mean that specific event is either interchangeable or has dependencies on other events, while still having a high impact when happening in “its preferred stage”.
The first mention to order impact on tumour phenotypes [ref] were performed in mice via overexpression of Ras analysis. More specifically, the oncogenic allele of Ras (H-RasG12V) and p53 pathway disruptor p53DD were studied. Order Ras-before-p53 resulted in highly malignant metastatic tumours, while the opposite occurrence of events led only to benignant cases.
\\

Derived works from this were performed [ref] on myeloprolifertive neoplasms allowing an mean quantity of somatic alterations of 5 to 10. Co-mutation of genes TET2 [ref] and JAK2 V617F [ref] were studied, based on a cohort using subjects holding both alterations. Based on which-happened-first approaches, different conditions properties were derived and associated to mutation order. JAK2-first patients being detected up to one decade  before TET2-first subjects suggested symptoms arise in different time-scales depending on alteration order acquisition. Besides, while stratifying patients thrombosis and polycythemia vera development were more likely for JAK-2 patients. This study concluded intrinsic epigenetic mechanisms caused by events taking place in a specific chronological order led to clones acquiring the same two mutations and showing distinct molecular and biological properties, which eventually resulted in divergent evolution of the disease. The properties —most likely malignant— of a clone is to reflect the sum of all the driver mutations acquired, therefore shadowing the chronological order approach. The general highlight of this scenario is that the initial mutation or whichever-came-first in the case we are analysing, can be responsible of an epigenetic landscape alteration, as TET2 does when it happens before JAK2. Due to the prevention of JAK2 of distorting the system down the line via up-regulation, the effect of such variance resulted in an attenuation of symptoms.
\\

Other scenarios [ref] contemplate breast cancer migration from breast (primary site) to bone (secondary site) using multi-compartment analysis. The mathematical modelling aims to show how the number of driver mutations and their order when altering cell’s metabolic processes are highly relevant for cancer progression. More precisely, compartment migration has been set to be extremely rare if the clones have not acquired the model-set alteration, i.e. wandering to the next stage requires a set of mutations, collected at any precedent phase. Eventually, the dynamics of the model reflect the effects of mutation order, and the contrast against a non-ordered approach. Still, an ordered modelling of the system introduces a strong hindrance in terms of cell compartment mobility, since cancer evolution is regulated by unordered dynamics.
\\

Four driver mutations {EPCAM, CD47, CD44, MET} are identified as significant in [ref]. EPCAM is targeted in the first compartment, which represents the location in the breast. CD47 and CD44 are located in the circulatory system, whose compartment represents the next level of the disease. Finally, the last step of the disease in this study leads to the bone. This scenario is triggered by MET driver mutation. The mutation order is proved not to be preferential, i.e. there’s no “correct” or preferred chronological occurrence of mutation events for the disease to evolve. However, the order of driver mutations is relevant for progression as the events arrangement sequentially unlock behavioural properties within the cell which lead “compartment evolution”. This rockets the combinatory of possible outcomes and prognosis whose main required input is a known chronological progression of the mutational events. [Figure Y]
\\

A particularity of this study is that CD47 and CD44 are both effective drivers whose order is irrelevant. This introduces the possibility of order interchangeability that does not alter evolution.
\\
\subsection{Chronological order results in diverse pathway dysregulation scenarios}
Continuing on the breast cancer scenario, [ref] also states driver mutations are mutations of specific genes responsible for the deregulation of pathways.
\\

A study \cite{Gerstung2011TheTumorigenesis} cover a wide collection of cancers to establish a chronological order among pathway mutation via a probabilistic graphical mode. It mainily relies on gene level order inference, later mapping of genes to pathways and transitively assorting pathway dysregulation. More precisely, the study focus on colorectal cancer, pancreatic cancer and primary glioblastoma to create an ordering of pathway dysregulation. All the specific assays first infer an event assorting for individual genes which later translate to pathway chronological order alteration, depending on particular gene belonging.\\

Putting everything together results in an unified model which acts as a pan-cancer maximal model, i.e. collects inferences from every special case and builds a general model. The generic approach agrees on Apoptosis-related pathways as the first dysregulation event occurring in the alteration, followed by TGF-b and KRAS signalling pathways, and G1/S phase transition control. \\
Also, events happening at the end of each chronology are are DNA damage control, WNT/Notch signalling, and JNK. All these, including Invasion-related pathways are found in the most advanced (late) stages.
One of the main conclusions is that pathway order constraints are way stronger than gene-level constraints.
The accuracy of the proposed (general) models is also restrained by gene networks operation which act at tissue and species level.
\\

Another study \cite{Cheng2012AGliomagenesis} follows the same approach: classification of the altered genes effects into the signalling pathways based on the mutations belonging. As an enhancement of a previous gene-focused approach \cite{Attolini2010ACancer}, this model assumes the pathway alteration happens when any of the members of the set of genes belonging to said pathway is altered. TP53, PI3KC1/AKT, PI3KC2, RAS, and RB signalling pathways in glioblastoma multiforme (GBM) disease are the main focus of the work.

The retinoblastoma (RB) signalling pathway is involved in proliferation, apoptosis and differentiation events of the cell \cite{Du2012TheTherapeutics}, and study's assays have identified it as an early-stage alteration event during gliomagenesis. Its disruption is consistent with G1/S checkpoint alteration, which eventually triggers oncogenesis \cite{Bertoli2013ControlPhases}.

TP53 mutation is also tagged early, since its malfunction is related to inability to control cell growth and proliferation.

The remaining assortments are established via \emph{in vitro} assays, where dependencies amongst events are set, e.g. AKT cannot initiate tumorgenesis until RAS alteration occurs.

Despite assuming independence between analyses using RESIC \cite{Attolini2010ACancer}, inconsistencies were not found in terms of chronological ordering, i.e. an event A happening before an event B in one analysis was contradicted in others (B before A).
\\

A particular finding in this approach is that the level of noise in gene-based temporal order analysis is higher than in the novel pathway-based approach, mainly because of the similar effects single gene alterations belonging to the same pathway have. This has been reflected in the form of weaker orderings in the pathway-based approach, in which 66,7\% of the orderings occur on a frequency >70\%, than in the gene-based one, in which frequency equals 40\%. Besides, ordering of genes has an extra flaw: theoretically true assortment can be shadowed by the fact multiple genes belong to the same pathway.

Also, important anomalies which could be clinically actionable arisen independently from pathway dysregulation can be hidden in pathway-driven ordering.

Last, an Subtype-specific approache \cite{Khakabimamaghani2019UncoveringDysregulation} is able to look at diverse progression orders, using Cancer Cell Fractions (CCF) somatic mutation data. First trials are made on synthetic data to benchmark its accuracy, and then a list of 14 driver mutation genes from colorectal adenocarcinoma (COAD) and 15 glioblastoma multiforme (GBM) were introduced.

Cancer subtypes are identified via clustering the pathway dysregulation orders of independent tumors, using tumor geneaology. Frequency or proportion of cancer cells present in samples are the core of the inference in that the higher is the recurrence, the stronger the order evidence.

This method allows grouping patients at pathway level based on trajectory, distinctly from phenotype subtyping (e.g. Breast cancer).


\subsection{Inheritance impact}
Inheritance of a single mutated gene might lead to several CRC outcomes [ref]
Since the dawn of high-throughput data, many inherited risk alleles have been identified across cancer subtypes, and increase the probability of developing a disease. Many Single nucleotide polymorphisms (SNPs) are nowadays largely known...
\\

It also depends on the time when the alteration has happened: before or after transmission, which does not exclude the probability of the offsprings of developing them naturally.