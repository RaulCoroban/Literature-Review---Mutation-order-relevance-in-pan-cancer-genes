Chronological order can be important:
Because sets an accumulation of consequences that lead the evolution of the disease.
\\

Order preference is not order restriction.
\\

Unlock cancerous properties at different stages, which can speed up some stage transitions or slow them down, all depending on cell’s biological age.
\\
There is a point where chronological order looses relevance, despite managing evolution: If the accumulated mutation set is enough to already converge to a final (metastatic?) stage in a short biological life-span, there is no use of knowing the order.
\\

Which states some mutations are interchangeable in terms of chronological order since they have the same effect and do not impact heavily in microenvironment and therefore progression.


Perhaps cancer labelling bound to anatomical position is applicable to generalise the pathological term, then the heterogeneity might seem large within a community. However, if we look at it from another perspective, special case existence, one per patient, becomes more obvious, and generalisation blurs, like only considering the leaves of an evolutionary tree.