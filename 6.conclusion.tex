SI EL ORDEN DE LAS MUTACIONES ES LAXO, ENTONCES NO HAY ORDEN.

After an analysis of pertinent studies we can conclude a chronological order of the mutational events can be important. It sets an accumulation of consequences that lead the evolution of the disease, generally starting with a driver mutation that eventually triggering a chain effect whose consequences are not only reflected at pathway level, but also confer further understanding on the condition. The awareness of this fact can serve as a motivation for neoplasms evolution and future predictions.
\\

Alteration order at a gene level has been proved as a major discovery in specific colorectal and breast cancer conditions, allowing a differentiation at outcome level within patients. It can highlight specific points of disruption within the system, which can clinically evolve to trug targets. However, due to the particular biological dynamics that regulates cell's system, and the variability each environment confers to it, they cannot conclude to a general model of order. Many events are labelled as interchangeable within the timeline, either because their impact is truly synonymous or because their ranking is ambiguous. This is the reason why we conclude there is an order \emph{preference} not and order \emph{chronological} restriction.
\\

Pathway-based approaches are also able to discern between disease aftermath and aid diagnosis. Starting from the gene temporal order organisation, any event of its timeline can dysregulate a full pathway. This situation can be amplified and disturb other pathways, then setting a new chain effect, although at a higher level. Such chronological setting is more solid as the gene-based way, as it sets an holistic view of the path-timeline. Still, progression is patient-specific. Having different circumstances might tire evolution towards different ends. Then chronological order is ultimately responsible for prognosis, although its navigation is totally susceptible to its surroundings.
\\

When inheritance enters the picture the order models built on clinical analysis are severed since they are order based. Germline mutations are events hard to track chronologically, in order to successfully include them in a temporal model, hence leaving a gap in the timeline, or obscuring assumptions which rely on clinical analysis only. They have the same impact as \emph{temporally unexpected} mutations: They unlock cancerous properties at different stages, which can speed up some stage transitions or slow them down, all depending on cell’s biological age.
\\

Perhaps cancer labelling on gene/pathway order terms rather than bound to anatomical position is applicable to generalise the pathological term, then the heterogeneity might seem large within a community. However, if we look at it from another perspective, special case existence, one per patient, becomes more obvious, and generalisation blurs, like only considering the leaves of an evolutionary tree.