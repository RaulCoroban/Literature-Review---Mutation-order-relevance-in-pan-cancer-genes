After an analysis of pertinent studies we can conclude a chronological order of the mutational events can be important. In most cases, the order can be reduced to as few as 2 events occurring in different order. This fact triggers an accumulation of consequences that lead the evolution of the disease, generally starting with a driver mutation that eventually starts a chain effect whose consequences are reflected at pathway level. The awareness of this fact can serve as a motivation for neoplasms evolution analysis and future predictions.
\\

Alteration order at a gene level has been proved as a major discovery in specific colorectal and breast cancer conditions, allowing a differentiation at outcome level within patients. It can highlight specific points of action within the progress of the disease, which can clinically evolve to trug targets. However, the order model can help distinguishing among patients as they evolve differently, therefore opening new sup-types. The cases supporting this fact are a majority, not a totality, which leaves space for other potential rare subtypes. This is the reason why we conclude there is an order \emph{preference} not and order \emph{chronological} restriction.
\\

Pathway-based approaches are also able to discern between disease aftermath and aid diagnosis. Starting from the gene temporal order organisation, any event of its timeline can dysregulate a full pathway. Such chronological setting is more solid than the gene-based way, as it sets an holistic view of the path-timeline which conserves the restrictions and properties already inferred at gene-level.

Understanding pathway alteration progression provides a general view of how a tumour evolves. Disruptions within the pathway take longer than gene mutations to become evident in its full reach, which makes them easier to map to critical points in tumour progress or even to condition stages.

Still, cancer dynamics remains patient-specific, where even cases under the same diagnosis can show an heterogeneous phenotype. Overall, pathway dysregulation offers a more holistic view of the scene, solving part of the drawbacks the gene order approach arises.
\\

When inheritance enters the picture, the order models built on clinical analysis can be questioned. Germline mutations confuse the timeline as events expected to be occurring late in the cancer progression could be found early.
They have the same impact as \emph{temporally unexpected} mutations: They unlock cancerous properties at different stages, which can speed up some stage transitions or interfere with a known condition progression by arising unexpected consequences.
