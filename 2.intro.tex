Cancer development is a progressive and complex multistep process within a system. The uncontrolled growth and cell division properties are general facts which have many hidden implications, both in terms of variables that condition its evolution and, more importantly, its consequences.
\\

Cancer evolution is usually driven by somatic sessions which lead to mutations that confer advantages to the affected cells. The final features of a cancerous cell are said to reveal an accumulation of the alterations acquired along the cell’s life.
\\

These altered cells, as a result of fitting mutations accumulation, eventually populate the environment and overtake its competitors (healthy individuals) presence via clonal expansions. Their fitness plays then an important role on its evolution and, therefore, survival.
\\

Many studies […] focus precisely on biomarkers in order to find a clear differentiating factor for specific cancer types. These results can lead to identifiable drug targets which to action clinically. Usually the most effective target markers are driver mutations. However, the efficiency of these methods in clinical trials is widely unknown and unpredictable. Drug low-response and patients relapse are common issues in the oncogenesis field, regularly related to the cancer’s heterogeneity. Cancerous cells evolution are highly dependent on patient’s genome and environment. Altogether, they build the niche where cancer differentiation, even within a subtype, is plausible. There is evidence that several individuals diagnosed under the same condition, have different evolution and prognosis […], mainly due to the potential assorted property of a cancerous tissue.
\\

\subsection{Chronological order}
Based on previous words about heterogeneity, multi-modal and divergent evolution scenarios are feasible, where dissimilar cancer populations arise. Each cell progeny might accumulate different mutations, or same mutations but in different order, respectively [Figure 1], which in turn obstructs treatment specificity, since factors such as evolution have been distorted.
In terms of causality, another factor which adds complexity to the system is the issue of discerning amongst mutations which happen due to the enhancing effect of another mutation(s) happening, or the ones which happen “naturally”, such as radiation, DNA replication mismatches, etc. If the first is confirmed (enhancement hints at causality) for a specific condition, tracing back which alterations already exist is trivial.
\\

As a more general first approach, cancer types can be identified based on molecular and genetic information, e.g. Breast cancer has 4 different known subtypes: Luminal A, Luminal B, Triple negative/basal-like and HER2-enriched, which accumulate different mutations, and therefore drive to divergent behaviours, evolution and prognosis.
\\

However, within the same subtype, populations can also be different in genomic terms and/or behavioural, i.e. different evolution compared to its peers [X]. Tumour development  rather evolves from an original common ancestral clone —which may or may not still be present in the environment— into a fully diverse cell community.
\\

Aversely, not only identical tumours share few gene alterations, but also show different mutation order acquisition. Then, does this chronological order [in which mutations happen] affect the evolution of the disease?
H0: Assuming a chronological order in mutation events, cancer evolution can be seen as a diagram of branched events, more or less probable to converge to the same state, depending on the mutation acquisition synergy effect.
\\

Classification, patient stratification or prediction can be done based on the stage’ history traceback analysis. 
\\

H1: Prognosis estimation can be based on time-point samples [ref] along during the case study. 
\\

H2: Early diagnostic and treatment to target the drivers would be delicate to calculate yet extremely precise if possible. 
\\

\subsection{Pathways}
A common belief is that a cell needs to acquire certain mutation-driven properties in order to become carcinogenic. Generally, these include a potential capacity of avoiding apoptosis, unlimited replication power, irresponsiveness to (external) signals, angiogenesis and other tissues invasion, among others \cite{Hanahan2011HallmarksGeneration} [Ref]. The regulation of such biological functions are bound to the cell’s pathway. Many point dysregulation events in these systems are related to cell’s malfunction.
\\

Picking up from the single-gene-mutation-like modelling above, where we focus on individual genes mutation, we cannot leave out further implications of this case: Gene mutation and consequent defective function has a cascade effect on the pathway it's related to. Studies like \cite{Gerstung2011TheTumorigenesis} found up to 70\% pathway frequency of alteration in all samples\footnote{These results might vary depending on the stage of the carcinomas}. Therefore, selective pressure ultimately acts at pathway level.

Nevertheless, and assuming cell signaling universality, pathway dysregulation might differ from one patient to another \cite{Ulitsky2010DEGAS:Diseases}. In fact, same cancer type diagnosed in different patients can result in vast heterogeneity of cases due to different dysregulation events trajectories (within same or different pathways) \cite{Khakabimamaghani2019UncoveringDysregulation}, i.e. the alteration events happening in different order among organisms can be one of the driving forces that induce such divergence.

A pathway-driven approach can be advantageous in that we can combine sporadic mutations a simple single event, which allows better comparison of functional consequences, in the same way profile-driven techniques help... sequence-sequence comparison \cite{Cheng2012AGliomagenesis}.
\\

Is this temporal order responsible for heterogeneous yet highly-similar pathways dysregulation among patients under the same diagnosis? [X]
Previous background of individuals, such as environment exposure (nutrition, smoking, ambient factors, etc.) [X] already sets different start points from which a condition can be developed, i.e. the genetic scenario influences an individual’s evolution, both as a healthy organism or as a diseased one.
\\

\subsection{Inheritance}

Notwithstanding, aside of somatic mutations, the genetic luggage we carry is also influential. Inherited mutations cause many family cancer syndromes, which make certain individuals more prone to develop a condition.
\\

Do inherited mutations have an impact? Is it larger than somatic mutations?
\\

H3: Inheritance acts as a barrier for traceback (H0), since heritage is commonly missing or hard to connect the offspring. Therefore prediction has to be made only on patient’s evolution, although historic might help foresee response, problems and evolution.