\subsection{Chronological order of mutation impact disease evolution}
We define the chronological order assuming any possible combination of event assortment. There is no golden order, although we hypothesise in terms of feasible outcomes, in which we rely on the existence of more or less favourable cases.
\\

Sufficient backing studies to confirm order are not yet available fully confirm a total order-driven system dynamic. However, some above projects have accurately identified the effects chronological order has in case evolution, which manifest the scenario is plausible. 
\\

In the same way the sequential occurrence is feasible, an unordered model (assumed up til now) can happen too. Furthermore, an hybrid system managing the dynamics of the clonal population can even find the balance among both. E.g. in [ref] we have observed order in CD47 and CD44 mutations is not relevant. Assuming a total-ordered model, this might be because their chronological order can be swapped without altering it. In a non-ordered model, the reason is trivial. Interestingly, the hybrid model (loosely ordered), accepts sequential order at some points, and gets more relaxed with synonymous ordered events. To simplify this estimation, we also assume no other interaction interfere with the model, albeit the scenario is far from probable. Event assorting evolution (and eventual branching) also opens up a discussion between linear and parallel metastasis progression models [Downloaded paper].
\\

Chronological order establishment is the main requirement to unleash the next stage  of oncologic studies which eventually leads to clinical application. Constituting this assortment then becomes the main challenge for the oncogenic field [X] based on the fact that common (overlapping) mutations amongst clones show vast clinical and histological differences. Said diversity might be due to divergent evolution within the neighbourhood —even sharing the same cancerous properties, accumulated alterations can still differ, and therefore not be relevant for the study— or due to multiple stage clone comparison —e.g. some cells have not reached metastasis yet and some do.
\\

However, in most cases, large heterogeneity is present amongst the clonal population which, via phylogenetic trees analysis, can present a starting scenario from which order can be retrieved [G] [Z]. Based on the studies above and curated knowledge [refs…], we know a full phylogenetic scenario displaying is challenging to retrieve, since the environment might contain a conglomerate of multiple divergent-evolved clones.

Rebuilding:
\cite{Attolini2010ACancer}

Phylogenetics approaches:
\cite{Beerenwinkel2005Mtreemix:Trees}, \cite{Rahnenfuhrer2005EstimatingScores}

CONTINUE ON THESE PAPERS:
\cite{Desper1999InferringData}, \cite{Desper2001Distance-basedOncogenesis}.

Networks:
\cite{Hjelm2006NewOncogenesis}, \cite{Gerstung2011TheTumorigenesis}

\\

Assuming historical order establishment, case classification, patient stratification and evolution prediction can rely on stage’ history traceback analysis. By knowing the progression of the case we could accurately diagnose and treat the disease based on evolution and not on a single (current) snapshot.\\

Pinpointing the exact stage of a cancer evolution having knowledge of past events leaves us with a highly sensitive model that helps us act on the most effective spot to break more lethal consequences.
\\

Besides, meta knowledge can be added to the study to aid chronological organisation, given the effects and evolution of mutations are known. For example, based on TNM staging system, clone starting node point evasion and tissue invasion is expected —although not limited to [Y]— in stage 4, therefore, effective driver mutations are expected later in the phylogenetic-tree-like model.
\\

Based on an evolutionary argument, there is also a “weighting” effect on each mutation: E.g. Apoptosis control can initiate cancer. The more the progeny grows, the higher is the probability of alterations acquisition which enter the game of survival [X]. Therefore, alterations driving apoptosis should be tagged as more risky in terms of prognosis. Radical-stage-inducing events such as apoptosis, cooperation loss, proliferation triggers, metastasis, etc. are highly relevant for the cause.
\\
\\

However, cancer complexity cannot be studied leaving interaction out. Clones build an established microenvironment in which, although might be lacking vital functions such as specific signal detection from external sources, communication amongst the members is still possible [A]. The sub clonal population is still present in the organism where the cancer developing, therefore this interaction must also be taken into consideration.


\subsection{Pathways}
Stronger evidence for order constraints in pathways than in gene level order is no surprise, since P acts as a system, while G allows order irrelevance.